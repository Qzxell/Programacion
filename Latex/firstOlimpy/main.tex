\documentclass[11pt]{scrartcl}

\usepackage[sexy]{evan}
\usepackage{url}
\usepackage{amsmath}

\author{Qzxell}
\title{Preparaci\'on para comenzar en la Olimpiadas}

\begin{document}
\maketitle
\section{Comenzando}
que hay de ti?

\section{Que son las Olimpiadas de Matematicas?}

Las Olimpiadas de Matem\'aticas son competiciones internacionales de matem\'aticas que re\'unen a estudiantes talentosos de todo el mundo. Estas competiciones tienen como objetivo fomentar el inter\'es y la pasi\'on por las matem\'aticas, as\'i como promover el desarrollo de habilidades de resoluci\'on de problemas y razonamiento l\'ogico.

Las Olimpiadas de Matemáticas suelen estar diseñadas para desafiar a los participantes con problemas matemáticos complejos y creativos que requieren un pensamiento analítico profundo y habilidades de razonamiento abstracto. Los problemas presentados en estas competiciones a menudo van más allá del currículo escolar y requieren un enfoque innovador y una perspectiva matemática fuera de lo común.

Existen varias razones por las cuales las Olimpiadas de Matemáticas son consideradas maravillosas:
\begin{itemize}
	\item Desarrollo de habilidades: Participar en las Olimpiadas de Matemáticas ayuda a desarrollar habilidades de resolución de problemas, pensamiento crítico, razonamiento lógico y creatividad matemática. Estas competencias son valiosas tanto en el ámbito académico como en el profesional.


	\item Fomento del talento: Las Olimpiadas de Matemáticas proporcionan una plataforma para que estudiantes talentosos en matemáticas muestren su habilidad y se destaquen. Estas competiciones ayudan a identificar y nutrir el talento matemático desde una edad temprana.


	\item Conexión global: Las Olimpiadas de Matemáticas reúnen a estudiantes de diferentes países y culturas, brindando una oportunidad para establecer conexiones internacionales, compartir conocimientos y experiencias, y fomentar la amistad y el entendimiento mutuo.

	\item Inspiración y motivación: Participar en las Olimpiadas de Matemáticas puede ser una fuente de inspiración para los estudiantes, mostrándoles el amplio mundo de las matemáticas y sus aplicaciones en diferentes campos. Estas competiciones pueden motivar a los participantes a seguir explorando y estudiando las matemáticas de manera profunda.
\end{itemize}
En resumen, las Olimpiadas de Matemáticas son maravillosas porque desafían a los estudiantes a superar límites, desarrollar habilidades matemáticas avanzadas y brindan una experiencia enriquecedora y gratificante que va más allá del currículo escolar tradicional.

\section{Las olimpiadas de Informatica IOI}
Si te interesa mas del mundo de la programaci\'on te dejo esta informacion.\cite{ioinformatics}

Las Olimpiadas Internacionales de Informática (IOI, por sus siglas en inglés) son una competición anual de programación y resolución de problemas en el campo de la informática. La IOI reúne a estudiantes talentosos de todo el mundo que tienen habilidades destacadas en programación y resolución de problemas algorítmicos.

Estas competiciones están diseñadas para desafiar a los participantes a resolver problemas complejos de programación utilizando sus habilidades de razonamiento lógico, algoritmos y estructuras de datos. Los problemas presentados en la IOI suelen ser desafiantes y requieren un pensamiento creativo y analítico para encontrar soluciones eficientes.

La IOI se lleva a cabo durante varios días y consta de múltiples rondas de problemas, donde los participantes trabajan individualmente para resolverlos en un tiempo determinado. Cada país envía a un equipo de estudiantes seleccionados a través de un riguroso proceso de selección interna.

Además de la competición en sí, la IOI también proporciona una plataforma para que los estudiantes interactúen con otros entusiastas de la informática de todo el mundo. Los participantes tienen la oportunidad de establecer conexiones internacionales, compartir conocimientos y experiencias, y participar en actividades educativas y culturales.

Las Olimpiadas Internacionales de Informática ofrecen a los estudiantes talentosos en informática la oportunidad de demostrar y desarrollar sus habilidades, así como de avanzar en su pasión por la programación y la resolución de problemas algorítmicos. Estas competiciones ayudan a fomentar el interés en la informática, promover el pensamiento computacional y formar a futuros líderes en el campo de la tecnología.

\section{Metas Claras, disiplina firme}
Al combinar metas claras con una disciplina firme, puedes aumentar tus posibilidades de éxito. Las metas claras te proporcionan un objetivo claro y la disciplina firme te ayuda a mantenerte en el camino correcto hacia ese objetivo, superando los desafíos y manteniendo la constancia en tus esfuerzos. Recuerda que la disciplina no siempre es fácil, pero es fundamental para lograr resultados significativos a largo plazo.

\section{Como comenzar?}
Querido hermano, esta meta no es facil en lo absoluto, en general nada es tan facil como te lo pintan, sin embargo, con una desici\'on firme y consisa puedes llegar a todo lo que te propones, con esto claro pasare a las instrucciones que me sirvieron para comenzar, siempre es mejor y m\'as eficaz trabajar con un entrenador, pero tu meta es una a largo plazo, \textbf{TIENES TIEMPO} por lo tanto, tienes que ser autodidacta. En esta guia me base mucho en el matematico Evan Chen te dejo su pagina aqui \cite{evanchen}, despues de visitarlo, dirigete asia a este link \url{https://web.evanchen.cc/faq-contest.html#C-0}   
	
Toda la informaci\'on valiosa que aun no llega a Per\'u esta en los angloparlantes, INGLES carajo , por lo tanto ve aprendiendo y acostumbrandote, y cualquier cosa que no encuentres, siempre asegurate de buscarlo en ingles.

Tambien todos libros que recomiendan estan en una libreria mega grande de internet.\cite{libgen}

\section{El resto}
Es practicar y practicar, que te mandare el temario y la guia en la que me base.

Donde puedes encontrar una comunidad mundiamente conocida que recomiendo para practicar es \cite{artofproblemsolving}, y ya si quieres programar te dejo esto tambien \url{https://aprende.olimpiada-informatica.org/cpp} es una guia completa en spanish para que te diviertas, puedes encontrar varios en internet, si quieres m\'as informacion no dudes en preguntar!!

att:Qzxell


\begin{thebibliography}{9}

\bibitem{artofproblemsolving}
Art of Problem Solving Community,
\emph{Art of Problem Solving Community},
[En línea]. Disponible en: \url{https://artofproblemsolving.com/community}
\bibitem{evanchen}
Chen, Evan.
\emph{Where to Start: Learning Resources for Olympiad Mathematics},
[En línea]. Disponible en: \url{https://web.evanchen.cc/wherestart.html}
\bibitem{libgen}
Libgen.is,
\emph{Library Genesis},
[En línea]. Disponible en: \url{https://libgen.is/}
\bibitem{youtube}
YouTube,
\emph{YouTube},
[En línea]. Disponible en: \url{https://youtube.com/}
\bibitem{ioinformatics}
International Olympiad in Informatics (IOI),
\emph{Getting Started},
[En línea]. Disponible en: \url{https://ioinformatics.org/page/getting-started/14}
\bibitem{vipulnaik}
Naik, Vipul.
\emph{Olympiad Resources},
[En línea]. Disponible en: \url{https://vipulnaik.com/olympiads/}

\end{thebibliography}

\end{document}
