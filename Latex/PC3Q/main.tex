\documentclass[11pt]{scrartcl}

\usepackage[sexy]{evan}
\usepackage{caption}
\usepackage{float}
\usepackage{tabularray}


\author{Axel Benjamin Rocca Cruz, 20234046A}
\title{Practica Calificada 3}
\begin{document}
\maketitle
\section{pregunta 1}
El equilibrio químico de la reacción $CO_{2} + H_{2}O$ a $H_{2}CO_{3}$ se puede representar en una gráfica de energía en función del avance de la reacción. Esta gráfica se conoce como diagrama de energía de reacción.

En el eje vertical de la gráfica se representa la energía, y en el eje horizontal se representa el avance de la reacción, que generalmente se expresa como un porcentaje de la reacción completa.

La reacción consiste en la formación de ácido carbónico (H2CO3) a partir de dióxido de carbono (CO2) y agua (H2O). Para representar la energía en la gráfica, consideremos los siguientes pasos:

    Energía de los reactivos: Representa la energía de los reactivos (CO2 y H2O) en la gráfica. Esta energía inicial se encuentra en un nivel más bajo.

    Energía de activación directa: Durante la reacción, se requiere una cierta cantidad de energía para que los enlaces entre los átomos de los reactivos se rompan y formen nuevos enlaces para dar lugar a los productos. La energía de activación directa se representa como una barrera energética en la gráfica. En este caso, la reacción directa de CO2 + H2O a H2CO3 puede tener una barrera energética relativamente alta.

    Energía de los productos: Una vez que se supera la barrera energética, la energía de los productos (H2CO3) se representa en un nivel más alto que la energía de los reactivos.

    Energía de activación inversa: Si la reacción es reversible, es decir, puede ir en ambas direcciones, también se representa una barrera energética en el sentido inverso (H2CO3 a CO2 + H2O). La energía de activación inversa puede tener un nivel similar o diferente al de la energía de activación directa.

    Energía de los productos en equilibrio: Cuando la reacción alcanza el equilibrio químico, la energía de los productos se estabiliza en un nivel determinado. En este caso, la energía de los productos en equilibrio está entre la energía de los reactivos y la energía de los productos individuales.
    \section{Pregunta 2}
    \begin{itemize}
	    \item \textbf{A 20 °C, ¿Cuántos gramos de KNO3 se disuelven en 10 mL de agua de densidad 1g/mL?} interpretando la grafica de KNO3 y su solubilidad a 20 °C en H2O es de 30:
	     

S = 30 g de KNO3 en 100 g de H2O

Siendo que la densidad del agua es de 1 g/ml 
\[
	100 mL \text{ de } H_{2}O -- 30 g \text{ de } KNO_{3}
\]
\[
	10 mL \text{ de } H_{2}O  --0.3 g \text{ de } KNO_{3}   
\]
Se disolverán hasta un máximo de 0.3 gramos de KNO3 en 10 mL de agua a una temperatura de 20 °C.
\item \textbf{A 50 $\circ C$, ¿Es posible disolver 80 g de KNO3 en 100 g de agua? }

	Podemos interpretar en la gr\'afica que la solubilidad del $KNO_{3}$ a 50 $\circ C$ en $H_{2}O$ es de 83 aproximadamente:
Por ende, realizamos los cálculos

S = 83 g de $KNO_{3}$ en 100 g de $H_{2}O$  
   
Por lo tanto, se puede disolver 80 g de $KNO_{3}$ en 100 g de agua, debido a que a la temperatura de $50 \circ C$ se puede disolver hasta un máximo de 83 g de $KNO_{3}$ en 100 g de agua.
\item \textbf{A 80 °C, ¿Es posible que 1,8 g de KNO3 estén disueltos en 1 mL de agua de densidad 1  g/mL?}

	Podemos interpretar en la gráfica que la solubilidad del KNO3 a 80 °C en H2O es de 170:
Por ende, realizamos los cálculos

S = 170 g de KNO3 en 100 g de H2O  

Siendo que la densidad del agua es de 1 g/ml 

100 mL de H2O  170 g de KNO3
1 mL de H2O  1,7 g de KNO3   

Por lo tanto, no se puede disolver 1,8 g de KNO3 en 1 mL de agua, debido a que a la temperatura de 80 °C se puede disolver hasta un máximo de 1,70 g de KNO3 en 1 mL de agua sin tener que cambiar las condiciones de la reacción. 
\item \textbf{A 90°C, si agrego 22 g de KNO3 a 10 g de agua y revuelvo con bagueta, ¿Cuántas fases se observarán?}

	Podemos interpretar en la gráfica que la solubilidad del KNO3 a 90 °C en H2O es de 200:
Por ende, realizamos los cálculos

S = 200 g de KNO3 en 100 g de H2O   

100 g de H2O -- 200 g de KNO3
10 de H2O  --20 g de KNO3   

Se observaran 2 fases, el precipitado solido del KNO3 que no se disolvió en el agua debido a que sobrepaso el máximo de solubilidad y  la fase liquida del KNO3 que si logro disolver en el agua.
\end{itemize}
\newpage
\section{Pregunta 3}
\begin{itemize}
	\item \textbf{Cual es el orden de reacci\'on?}
\begin{table}[H]
\caption{orden de las reacciones}  {\label{tab:1}}
\centering
\begin{tblr}{c c c c}
	\hline
	t(s) & [A] & ln [A] & 1/[A] \\ \hline
	0     & 0,600      & -0,510 &  1,666  \\
	100   & 0,497      &  -0,699 &   2,012 \\
	200   & 0,413      & -0,884   & 2,421 \\    
	300   & 0,344      &  -1,067  & 2,906 \\   
	400   & 0,285      &   -1,255 & 3,508\\  
	600   & 0,198      &  -1,619  & 5,050\\
	1000  & 0,094      &   -2,364 & 10,638\\ 
\end{tblr}
\end{table}
\item \textbf{Cuál es el valor de la constante de velocidad, k?}

tenemos que hallar  $\tan \theta$

\[
	\tan \theta = \dfrac{ln [A]_{2}-ln [A]_{1}}{t_{2}-t{1}} 
\]
\[
	\tan \theta = \dfrac{-2,364 + 1,255}{1000-400=-1,84 \cdot 10^{-3}}
\]
entonces $-m=K$ , donde m es la pendiente, que es equivalente a $\tan \theta$
\item \textbf{Cuál es el valor de [A] a t= 750 s?}

	\[
		ln[A]_{t} =-Kt +ln[A]_{0}
	\]
	\[
		ln[A]_{750}=-(1,84\cdot 10^{-3})(750)+(-0,510)
	\]
	\[
	ln[A]_{750}=0,151M
	\]
\end{itemize}





\section{Pregunta 4} Para calcular la molalidad, necesitamos conocer la masa del soluto (etanol) y la masa del solvente (agua).

En primer lugar, vamos a calcular la masa del etanol y del agua en la solución a 15°C.

Supongamos que tenemos 100 g de solución. Como la solución tiene un 10\% en masa de etanol, la masa de etanol sería el 10\% de 100 g, es decir, 10 g. La masa de agua sería entonces el complemento: 100 g - 10 g = 90 g.

Ahora vamos a calcular la masa del etanol y del agua en la solución a 25°C.

Supongamos que tenemos nuevamente 100 g de solución. La masa de etanol sigue siendo el 10\% de 100 g, es decir, 10 g. La masa de agua sería el complemento: 100 g - 10 g = 90 g.

Ahora tenemos las masas del etanol y del agua en ambas temperaturas.

Para calcular la molalidad, necesitamos convertir las masas en moles. Para eso, necesitamos conocer las masas molares del etanol y del agua.

La masa molar del etanol (C2H5OH) es:
12.01 g/mol (C) + 2 * 1.01 g/mol (H) + 16.00 g/mol (O) + 12.01 g/mol (C) + 1.01 g/mol (H) + 1.01 g/mol (H) = 46.08 g/mol

La masa molar del agua (H2O) es:
2 * 1.01 g/mol (H) + 16.00 g/mol (O) = 18.02 g/mol

Ahora podemos calcular la molalidad en cada temperatura.
\begin{itemize}
	\item A 15°C:
Molalidad = (moles de soluto) / (kilogramos de solvente)

Moles de soluto (etanol) = masa de etanol / masa molar del etanol
Moles de soluto = 10 g / 46.08 g/mol = 0.217 mol

Kilogramos de solvente (agua) = masa de agua / 1000
Kilogramos de solvente = 90 g / 1000 = 0.09 kg

Molalidad = 0.217 mol / 0.09 kg = 2.41 mol/kg

\item A 25°C:
Molalidad = (moles de soluto) / (kilogramos de solvente)

Moles de soluto (etanol) = masa de etanol / masa molar del etanol
Moles de soluto = 10 g / 46.08 g/mol = 0.217 mol

Kilogramos de solvente (agua) = masa de agua / 1000
Kilogramos de solvente = 90 g / 1000 = 0.09 kg

Molalidad = 0.217 mol / 0.09 kg = 2.41 mol/kg

Entonces, la molalidad en ambas temperaturas es aproximadamente 2.41 mol/kg.
\end{itemize}
\section{Pregunta 5}
Dado que la cinética de la reacción global muestra que es de segundo orden en NO y de primer orden en H2, es necesario identificar una etapa de reacción que sea coherente con esta velocidad. En este caso, la única opción viable es la segunda etapa, ya que implica una especie con una cinética de primer orden en la reacción.
  
de las siguientes reacciones r\'apidas : 
\[
	2NO_{(g)} \rightleftarrows N_{2} O_{2(g)}
\]
\[
	N_{2} O_{(g)} \rightleftarrows N_{2(g)} + H_{2}O_{(g)}
\]
Se identifica la etapa intermedia que conecta las etapas de reacción de rápida cinética. En este caso, la etapa intermedia se encuentra en la etapa 2, donde ocurre la formación de N2O2 en la primera etapa y su posterior consumo en la tercera etapa.
\[
	N_{2} O_{2(g)} + H_{2(g)}\rightleftarrows 2NO_{(g)}  + H_{2}O_{(g)}
\]
al sumar las 3 equaciones se obtiene la ecuaci\'on global.
\[
	2NO_{(g)} + 2H_{2(g)} \rightleftarrows N_{2(g)} +2H_{2} O_{(g)}
\]

La velocidad de reacción en la etapa 2 está influenciada por la concentración de $ [ N_{2}O_{2} ] $. Debido a que la concentración de $[N_{2}O_{2}]$ es directamente proporcional a la concentración de $[NO]^2$, se confirma que la reacción es de segundo orden con respecto a [NO].

\textbf{En resumen}, el mecanismo propuesto, donde la etapa 2 actúa como la etapa de velocidad limitante, concuerda con el orden experimental de reacción de segundo orden en [NO] y de primer orden en $[H_{2}]$.

\end{document}

