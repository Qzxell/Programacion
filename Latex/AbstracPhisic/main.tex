\documentclass{book}
\usepackage{amsmath}

\author{ Qzxell}
\title{Resumen de los temas de Fisica}

\begin{document}
\maketitle
\tableofcontents
\chapter{D\'inamica de una p\'articula}
\section{Movimiento curvilineo}
dado que $F=ma$, y que las aceleraciones tangenciales y centripetas son. 
\[ a_{t}=\frac{dv}{dt}\]
\[a_{N}=\frac{v^{2}}{p} \]
entonces se puede obtener que:
\[F_{N}=\dfrac{mv^{2}}{p}\]
donde p : es el radio de curvatura, en el caso de movimiento circular , p = radio de la circunferencia, y como $a=w \times v=w \times w \times r$
\[
F_{N}=mw^{2}R
\]
para el caso de circular uniforme, la unica fuerza es la centripeta, por lo tanto:
\[
F=ma=mw \times v=w\times (mv)=w\times p
\]

\sebsection{Momentum Angular}








\end{document}
